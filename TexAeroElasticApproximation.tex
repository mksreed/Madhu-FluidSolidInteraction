\documentclass{article}
\title{Aeroelastic Approximation}
\author{Madhu Sreedharl}
\begin{document}

 So we have been exploring the space
of $U_R$ dimensionless  parameters, to see if some approximation allowed
building simple enough models. This approach has been fruitful in
the limit of very low reduced velocities. The fluid could then be
considered as still, with regard to the dynamics of the solid. As a consequence, and in the limit of small motion of the solid,
we obtain very general results on the form of the interaction force
exerted by the fluid on the solid. We found fluid-added stiffness,
fluid-added mass, fluid-added damping, more complex history effects of viscosity,
and even coupling with fluid modes. These models are quite useful in practice, in off-shore engineering, in ship design,
or biomechanics, just to cite a few. But this limit situation clearly does
not cover all possible interactions. Let us now explore the effect of flow.

 Look at the case of vibration of
a plane tail as illustrated here. The fluid velocity across
the wing is certainly not small. Let us see if we can build some model for
this, and many other phenomena that would
be related to the effect of flow. 

 Conversely to the previous approximation,
and quite symmetrically, in fact, we now explore the case where
the reduced velocity is very large. Remember that the reduced velocity
is the ratio of the two time scales, fluid and solid. A very large reduced velocity
means that the time scale of the solid dynamics is much longer
than that of the fluid dynamics. In other terms, the time of oscillation or of wave propagation in the solid is
now much longer than the time of convection of a free particle
across the length scale L. What does it mean now? Let us imagine now that we follow in time
the evolution of a quantity pertaining to the solid. For instance,
the displacement of a material point. This quantity evolves with a long
time scale that I call $T_{solid}$. Conversely, a quantity pertaining to
the fluid dynamics, such as the position of a fluid particle, evolves on a much
shorter time scale, $T_{fluid}$. Intuitively, we may imagine
that there is a limit when these time scales are very different, such that the dynamics of the fluid 
occurs in interaction with a fixed solid. This would certainly a  be much more
simple framework to model fluid-solid interactions if the solid
does not seem to move. Note that by stating
that the solid is fixed, I do not mean that it does not move. I just say that its own motion occurs so slowly that I can neglect it
when I'm computing the motion of the fluid. But of course I will also
have to compute it at some point. So we are looking for
the framework of an approximation that is exactly the reverse of
that of the previous lecturses. Instead of a non-moving fluid,
it is a non-moving solid. As an example, let us consider
the oscillating wing that I showed before. The flow velocity in the reference
frame of the wing is of the order of a hundred meters per second, say. So that is $T_{fluid}$ is the order of .01 second for wing of about a meter of  chord. Then one oscillation of the wing
occurs in about one second. Consequently the reduced velocity,
which is a ratio between one second and .01 second, is about 100. Which is not very large, but
might be large enough to do something. 

All I've just said here was essentially
an intuition that if the two time scales are so separated, there is
going to be a more solvable problem. Let us see that in
the equations themselves. The first step is to have a good
choice of the dimensionless numbers. The ones we choose originally
where the Reynolds number, the Froude number, and the Cauchy number. They're all based on
the scalar velocity $U_0$. This is perfectly adequate. And there is no need
to change them this time. 
In o$U_R$ equations it is more appropriate
to use the dimensionless time based on the fluid dynamics, not on
the solid dynamics which is so slow. So we have t tilde equal t over $T_{fluid}$ where $T_{fluid}$ is L over $U_0$,
A time scale of 1 in t tilde is appropriate for the
description of the dynamics of the fluid. Finally, we shall use
dimensionless velocities and press$U_R$es based on $U_0$.  $$\tilde{U}=\frac{U}{U_0} \quad and  \quad  \tilde{P} = \frac{P}{
\rho {U_0}^2}$$ 

 Now, let us write down the dimensionless
equations using these variables. To do this, we only need to
replace the dimensional variables by the dimensionless ones in
the original dimensional equations. We have already done this kind of change
of variable in the previous part of the course, when we explored the case
of a very low reduced velocity. So for the fluid we have the mass balance
and the momentum balance as before. On the solid side,
I just state here that the displacement can be represented using
a single mode approximation. For this mode, I use the dimensionless
oscillator equation. At the interface,
I write the kinematic condition and the dynamic condition
in dimensionless forms. They state that the velocities
are continuous at the interface, for the first one. And that the force that applies
on the mode $f_ FS$ is the projection of the local
fluid loadings on the modal shape $\phi$. Now that we have these equations,
we should be able, somehow, to see how $U_R$ assumption of a very high reduced
velocity would simplify the equations. 

Again, the reduced velocity is going to
appear in the scaling of the boundary conditions on the fluid domain. The scalar $U_0$ corresponds to
the reference velocity of the fluid, which I can symbolically write as
a boundary condition on the fluid. The velocity there scales as $U_0$ so
that in the dimensionless value it is of the order of $U_0$
over $U_0$, which is 1. At the fluid solid interface
the kinematic condition is that the fluid velocity is
equal to the solid velocity. What is the magnitude
of the solid velocity? We know that the solid displacement
scales as 
$\xi_0$ and that it evolves  in a time scale 
$T_{solid}$. This means that the dimensionless displacements scale as 
$\frac{\xi_0}{L}$. Which is exactly
the displacement number D. This displacement evolves in a dimensionless
timescale of $T_{solid}$ over $T_{fluid}$, which is precisely the reduced velocity,
$U_R$. So to summarize, $\overline{\xi}$ is of order D and it varies of a time scale of $U_R$. Then we can say that the order of 
$\frac{d\overline{\xi}}{dt}$   is $\frac{D}{U_R}$. So the order of the velocity in the fluid at the interface is also D over $U_R$. [MUSIC] Let us summarize. The dynamics of the fluid is governed, on
one hand, by a condition of order one and on the other end by a condition
of order D over $U_R$ as interface. Certainly if $U_R$ is much
larger than D the second condition can be set to zero
without changing much the result. This correspond exactly to what we
meant by neglecting the solid dynamics. The solid moves so
slowly that its velocity is not expected to play any role in the fluid dynamics. 

 In the general case, we have an interface
between the fluid and the solid, which has a deformation and a velocity. In the approximation that we just built
it has deformation but no velocity. This defines the quasi-static
aeroelasticity approximation. The motion of the solid is, from the point
of view from the fluid, Quasi Static. We shall say that the position of
the interface is frozen in time. 

 So, in the general case,
the fluid dynamics and the solid dynamics are coupled
by the kinematic and dynamic conditions, at interface,
and they evolve simultaneously. But in our approximation
we have two dynamics. One slow, which is that of the solid, and
one fast, which is that of the fluid. The solid dynamics gives the position
of the interface through a kinematic condition that is considered as time
independent for the fluid dynamics. This means that we're back to
classical problem of fluid dynamics with a rigid boundary. And we know a lot about that. We have plenty of results and methods
from theory, experiments, computations. Then the solution of the fluid dynamics
gives a load at the interface. With that we can compute
a solid dynamics and so on. So the two dynamics are coupled, but because we have a separation of time
scales, we do not have to solve them simultaneously, and
this is a major result. In other terms, if we go back to our
example of a wing oscillating in a flow this means that we have broken
the problem of the interaction as a succession of problems of
flow around a fixed deformed wing plus the problem of vibration. This seems much simpler. And it is. To summarize, we have found that for
 very large reduced velocities, we can build a model where the velocity of
the solid at the  interface can be neglected. And the key result is that
we can use all we know in fluid mechanics with rigid boundaries
to predict the motion of the interface. Next we shall see if such
an approximation allows predicting flow induced motions
such as the example we showed. 

\end{document}